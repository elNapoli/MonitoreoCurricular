\subsection*{Anexo A: Traducción a español extracto Ley de consumo (Ley nº 8.078/90).}

En el caso brasileño, la ley de consumo (Ley nº 8.078/90) define como consumidor y proveedor:
"Art. 2º: Consumidor es toda persona física o jurídica que adquiere o utiliza un producto o servicio como destinatario final."
"Art. 3º: Proveedor es toda persona física o jurídica, pública o privada, nacional o extranjera, así como los entes despersonalizados, que desarrollan actividades de producción, montaje, creación, construcción, transformación, importación, exportación, distribución o comercialización de productos o servicios."\\

Según la ley de consumo en Brasil, el proveedor debe indemnizar independiente de culpa, es la llamada responsabilidad objetiva del proveedor. Así dispone el artículo 14 de la ley 8.078/90: "El proveedor de servicios responde, independientemente de la existencia de culpa, por la reparación de los daños causados a los consumidores por defectos relativos a la prestación de los servicios, así como por informaciones insuficientes o inadecuadas sobre el uso y riesgos."\\

Muchos tribunales a lo largo del territorio brasileño han unificado el entendimiento en sus decisiones sobre la reparación por espera en la fila de atención.\\

Los jueces del Tribunal de Mínima Cuantía del Estado de Paraná, incluso han editado un enunciado que expresa la posición de esta corte sobre el caso:
Enunciado N. º 2.7– Fila de banco – daño moral: El tiempo de espera en la fila de una sucursal bancaria, en tiempo excesivo, caracteriza falla en la prestación del servicio y ocasiona reparación por daños morales.\\

La jurisprudencia de la Corte de Apelaciones del Estado de Sergipe apunta:
APELACIÓN CIVIL. ACCIÓN DE INDEMNIZACIÓN POR DAÑOS MORALES ORIGINARIO DE LA ESPERA PARA ATENCIÓN EN FILA DE SUCURSAL BANCARIA. DEMORA INJUSTIFICADA EN LA ATENCIÓN. RESPONSABILIDADE CIVIL CONFIGURADA. DESÍDIA QUE AFRONTA LA DIGNIDADE DE LA PERSONA HUMANA. DANO MORAL SUSCEPTIBLE DE REPARACIÓN PECUNIÁRIA. SÚMULA Nº 04, DO TJSE. La larga espera en fila de sucursal bancaria, además del límite temporal impuesto por la ley municipal, es un hecho generador de desgaste físico y emocional, capaz de afectar la honra subjetiva de la persona y atingir un derecho inmaterial suyo, originando, por lo tanto, daño moral susceptible de reparación pecuniaria. Reforma de la sentencia, para condenar el demandado en el pago de indemnización en el valor de R\$ 1.000,00 (mil reales), a título de daños no patrimoniales, con interés de 1\% (un por ciento) mensuales, desde la fecha del evento (29/08/2012) y corrección monetaria por el INPC, a partir de esta fecha (Corte de Apelaciones de Sergipe - AC 201300226617; Ac. 678/2014; Segunda Cámara Civil; Rel. Des. Cezário Siqueira Neto; Julg. 17/02/2014; DJSE 11/03/2014).\\
