\subsection*{Anexo C: Sketch nodo de sensores.}
\addcontentsline{toc}{subsection}{Anexo C: Sketch nodo de sensores.}
\label{anexo-c}

\hbox{}

\begin{verbatim}

#include <SoftwareSerial.h>
#include <TinyGPS.h>
#include <Wire.h>
#include <XBee.h>


#define RXPIN 3
#define TXPIN 5
#define GPSBAUD 4800

TinyGPS gps;
SoftwareSerial uart_gps(RXPIN, TXPIN);

void getgps(TinyGPS &gps);

int HMC6352SlaveAddress = 0x42; 
int HMC6352ReadAddress = 0x41; //"A" in hex, A command is: 
float direccion, latitud, longitud = 0;

uint8_t text[10] = {0, 0, 0, 0, 0, 0, 0, 0, 0, 0};
uint8_t tramaBoton[2] = { 1, 1 };
XBee xbee = XBee();
Tx16Request tx = Tx16Request(0x1000, text, sizeof(text));
Tx16Request txBoton = Tx16Request(0x1000, tramaBoton, sizeof(tramaBoton));

const int buttonPin = 2;
int botonValue = 0;

int banderaSend = 0;


void setup(){
  pinMode(buttonPin,INPUT);

  HMC6352SlaveAddress = HMC6352SlaveAddress >> 1;

  Serial.begin(9600);
  uart_gps.begin(GPSBAUD);
  Wire.begin();
  
  Serial.println("waiting for signal");
}

void loop(){
  botonValue = digitalRead(buttonPin);  // read input value
  if (botonValue == HIGH) {         
    xbee.send(txBoton); 
    banderaSend = 1;
    delay(20000);
  }

  //Read the serial port to see if GPS data is available
  while(uart_gps.available() && banderaSend !=0){
      byte c = uart_gps.read();
      //If incoming data is GPS data, process it
      if(gps.encode(c)){
        gps.f_get_position(&latitud, &longitud);
        float direccion = compass();
        if (latitud != 0 && longitud != 0){
          sending(latitud,longitud,direccion);
          delay(30000);
        }
      }
  }
}

void sending(float lat, float lon, double grados){
  double enteraLon, fraccLon, enteraLat, fraccLat;
  int banderaSign;

  //linda idea para ahorrar un espacio del array
  //flag que indica signos de coordenadas y si se pasa de 255º la brujula
  //saludos a las limitaciones de los xbee
  // (o mas bien a los creadores del protocolo)
  //uint8_t = 0 a +255

  if (lat < 0 && lon < 0){ // ambos negativos
    banderaSign = 0;
  } else if (lat > 0 && lon < 0){ // lat pos, lon neg
    banderaSign = 1;
  } else if (lat < 0 && lon > 0){ // lat neg, lon pos
    banderaSign = 2;
  } else {                        //ambos positivos
    banderaSign = 3;
  }
  int direcc = int(grados);
  if (direcc > 255){
    banderaSign = banderaSign + 10;
    direcc = direcc - 255;
  }

  fraccLon = modf(lon, &enteraLon);
  fraccLat = modf(lat, &enteraLat);

  String dLon = String(int(abs(fraccLon*10000)));
  String dLat = String(int(abs(fraccLat*10000)));

  text[0] = banderaSign;
  text[1] = int(abs(enteraLon));
  text[2] = dLon.substring(0,2).toInt();
  text[3] = dLon.substring(2,4).toInt();
  text[4] = int(abs(enteraLat));
  text[5] = dLat.substring(0,2).toInt();
  text[6] = dLat.substring(2,4).toInt();
  text[7] = direcc;
  xbee.send(tx);
}

float compass(){
  //"Get Data. Compensate and Calculate New Heading"
  Wire.beginTransmission(HMC6352SlaveAddress);
  Wire.write(HMC6352ReadAddress);              // The "Get Data" command
  Wire.endTransmission();

  //time delays required by HMC6352 upon receipt of the command
  //Get Data. Compensate and Calculate New Heading : 6ms
  delay(6);

  Wire.requestFrom(HMC6352SlaveAddress, 2); 

  //"The heading output data will be the value in tenths of degrees
  //from zero to 3599 and provided in binary format over the two bytes."
  byte MSB = Wire.read();
  byte LSB = Wire.read();

  float headingSum = (MSB << 8) + LSB; //(MSB / LSB sum)
  float headingInt = headingSum / 10; 

  return headingInt;
}

\end{verbatim}
