
\subsection{Conslusiones}

El objetivo general de implementar un sistema capaz de solicitar un número de atención a través de una aplicación móvil y realizar el seguimiento de éste utilizando notififcaciones push para \textit{Android}, se ha cumplido en su totalidad. Lo anterior queda en evidencia en el capítulo 6.\\

El análisis previo de las tecnologías disponibles permitió al alumno interiorizarse y elegir adecuadamente las mejores opciones para  construir el sistema. Específicamente, en esta etapa es vital responder las preguntas ¿Qué herramientas utilizar? y ¿Cómo comunicar las diferentes piezas de software que formarán el sistema utilizando las tecnologías elegidas?. Resolver dichas interrogantes, ayuda a consolidar una base sólida para desarrollar sin mayores inconvenientes el prototipo y no encontrarse con problemas graves a la mitad de la realización del proyecto. Cometer estos errores provocan desenlaces que significan un desaprovechamiento importante de tiempo y en  ocasiones comprometen recursos económicos.\\

A lo anterior se suma el hecho de lograr correctamente las etapas de análisis y diseño. Para ello, son clave las reuniones iniciales con el cliente o profesor patrocinante. De esta forma, el alumno tiene total claridad de los requisitos del sistema a desarrollar y de los acuerdos tomados para encaminar la solución que se propone a la problemática dada.\\

La solución que propone este proyecto de titulación presenta una oportunidad real para mejorar la calidad de vida de las personas. De tal forma que los tiempos excesivos de espera, cada día tienden a aumentar y son hechos generadores de desgaste físico y emocional que son capaces de afectar la honra de una persona, sean transformados o disfrazados como una oportunidad para que los clientes aprovechen mejor su tiempo cuando van a realizar una compra o un trámite, pudiendo acudir a otras depencencias y regresar cuando la aplicación le avise que ya se acerca su turno para ser atendido.\\

Esta solución calza perfectamente en el paradigma de internet de las cosas y también en el contexto de ciudades inteligentes. Día a día se van integrando más artículos cotidianos que antes era impensable que se puedan comunicar con otros dispositivos, o con internet mismo, incluso no se pensaba en que podían existir. Hoy contamos con la presencia de relojes inteligentes que cuentan con diversas características que desbordan el marco básico que define a un reloj análogo y, que en la medida que su hardware lo permita podemos insertar diversas funcionalidades que estarán disponibles en nuestra muñeca.\\


\subsection{Trabajo futuro}

El paso siguiente es incluir esta solución al sistema actual que poseen los centros de atención. Contar con la posibilidad de integrarlo con el sistema rudimentario de los tickets de papel, es una ventaja para cualquier centro pues contaría con un sistema de administración de filas de espera que incluye a todos sus clientes y no discrimina a quienes no tienen un celular inteligente o plan de datos.\\

Otro punto importante, es proponer un modelo de negocios para este sistema. Este proyecto de título fácilmente puede ser vendido como un servicio central en el que diversas plataformas pueden conectarse a éste y así informar a sus clientes sobre el estado de su turno de atención.\\