\subsection{Planificación}
	\subsubsection{Metodología}
	Todo desarrollo de software es riesgoso y difícil de controlar, pero si no llevamos una metodología de por medio, se obtiene clientes insatisfechos con el resultado y desarrolladores aun mas. 
	\\
	
	Las metodologías de desarrollo son como "recetas de cocina, en el que se van indicando paso a paso todas las actividades a realizar para lograr el producto informático deseado, indicando además qué personas deben participar en el desarrollo de las actividades y qué papel deben de tener\cite{MET06}.
	\\
	
	El ciclo de vida utilizado para el desarrollo de este proyecto es el de Entrega incremental. En un proceso de desarrollo incremental, los clientes identifican, a grandes rasgos, los servicios que proporcionará el sistema, se identifican cuáles son los servicios más y menos importantes, entonces, se definen varios incrementos en donde cada uno proporciona un subconjunto de la funcionalidad del sistema.
	
	

	
	
	
	
	\subsubsection{Plan de trabajo}
	\subsubsection{Validación}
\subsection{Análisis}
	\subsubsection{Requerimientos}
	\subsubsection{Modelo conceptual}
	\subsubsection{Casos de uso}
\subsection{Diseño e implementación}
	\subsubsection{Diagramas de componentes}
	\subsubsection{Modelo de datos}
	\subsubsection{Módulo historial curricular}
	\subsubsection{Metodología}
	
\subsection{Validación del software}
