
los sistemas de información 

El objetivo general de diseñar y construir  un prototipo de una plataforma web que apoye al monitoreo curricular de pregrado, se ha cumplido en su totalidad. Lo anterior queda en evidencia en el capítulo \ref{Desarrollo_Sistema}.
\\

\subsubsection{Trabajo futuro}
	
El paso siguiente es la integración total del software desarrollado con los sistemas de la Universidad, para ello se elaboró una propuesta de trabajo, la cual se puede ver en la Tabla \ref{Tabla_Propuesta_trabajo}.


	\begin{longtable}{l|p{7cm} |p{4cm}}
		
		\caption{Propuesta de trabajo}
		\label{Tabla_Propuesta_trabajo}\\
		
		
		\hline
		\endfirsthead
		\multicolumn{3}{c}%
		{\tablename\ \thetable\ -- \textit{Continuación de la pagina anterior}} \\
		\hline
		
		\hline
		\endhead
		\hline \multicolumn{3}{r}{\textit{Continúa en la página siguiente}} \\
		\endfoot
		\hline
		\endlastfoot
		
		\rowcolor{LightBlue2} Nombre hito & Descripción & Actores\\ \hline
		Reunión &
		Coordinar una reunión con Don Milton Muñoz para  crear un ambiente de desarrollo con el fin de que el alumno tenga acceso a los datos reales de la universidad & 
		Baldomero Águila, Mauricio Ruiz-Tagle, Milton Muñoz\\ \hline
		
		Integración &
		El alumno tesista tiene que integrar en su totalidad el sistema desarrollado en el ambiente de desarrollo & 
		Baldomero Águila.\\ \hline
		
		Reunión &
		Coordinar una reunión formar con Registro académico, con el fin de que este departamento provea documentación referente a los cambios curriculares & 
		Baldomero Águila, Mauricio Ruiz-Tagle, Cristina Barriga y Pilar Alarcón.\\ \hline
		
		Documentación &
		Toda la información entregada por registro académico, se debe entregar en formato digital, en caso contrario, se deben escanear todos los documentos & 
		Alumno ayudante.\\ \hline
		
		
		Poblar BD &
		Se deben subir todos los documentos facilitados al sistema con el fin de empezar a registrar los historiales curriculares de todas las facultades.& 
		Baldomero Águila, Pilar Alarcón.\\ 

		\hline \hline
	
	\end{longtable}
	
	
	En cuanto al alcance del proyecto, el prototipo desarrollado podría ser mejorado incluyendo un módulo estadístico, en el cual se pueda ver de forma gráfica y clara algunas variables, como por ejemplo: las carreras con más cambios curriculares, el porcentaje de carreras innovada, entre otras variables que se pueden considerar importantes al momento de la generación de informes que apoyen procesos estratégicos de seguimiento.
	\\
	
	Finalmente, otra mejora que se podría hacer al prototipo, es aumentar el tipo de archivos soportado, puesto que como la información referente a los cambios curriculares se encuentra en su gran mayoría de forma física, ésta información debe ser escaneada antes de ser subida, y generalmente los software que escanean guardan los documentos en formato imagen y actualmente el software solo permite archivos en formato PDF, por lo que esta mejora supone una disminución de tiempo al momento de querer subir un documento a la plataforma.