
			"La Universidad Austral de Chile es una institución acreditada que forma profesionales y graduados de pre 
			y postgrado, con un sello caracterizado por la excelencia académica, el compromiso con la libertad y con 
			el medio sociocultural, el respeto por la diversidad, la responsabilidad social, entre otros"\cite{MOD07}. 
			El cumplimiento de estas definiciones establecidas en el \textit{ modelo educacional y enfoque curricular}, requieren, entre otros, de procesos internos de la organización que apoyen la gestión educativa, en particular, de pregrado. Una de las funcionalidades más importantes en este ámbito, es la gestión de los proyectos curriculares de la carrera. Por esto es de gran importancia el conocer el historial curricular de cada carrera, necesidad que da origen al presente proyecto.
			\\
			
			El objetivo principal del presente proyecto de tesis consiste en diseñar y desarrollar un prototipo de 
			plataforma web que permita gestionar el historial curricular de cada carrera de la Universidad Austral 
			de Chile, el cual permitirá a distintas unidades de la universidad tener una mejor información curricular 
			de las carreras y así facilitar el trabajo que día a día realizan.
			\\

			
			El sistema web se desarrollará para las dependencias de la Universidad Austral de Chile, es por eso mismo que
			el alumno tesista debe adaptarse a las tecnologías que la universidad utiliza, por esta razón la solución se 
			desarrollará en las siguiente tecnologías: Microsoft Visual Studio 2013, Microsoft SQL Server 2008 
			(solo en ambiente de desarrollo, una vez finalizado el proyecto se migrará a SyBase, el cual es el motor 
			de base de datos que utiliza la universidad), Visual Basic como lenguaje del servidor, JavaScript, css3, HTML5 
			y GitHub como gestor de versiones.
			\\

			
			El sistema web beneficiará a los departamentos del área de pregrado de la Universidad en cual están 
			constantemente manipulando información curricular de las carreras, estos departamento son los siguientes: 
			Departamento de Aseguramiento de la Calidad e Innovación Curricular (DACIC), Departamento de Registro 
			Académico Estudiantil y  Departamento de Admisión y Matricula, además beneficiará a la propia escuela, ya que les permitirá contar con
			información histórica curricular.
			\\
			
			Las ventajas de contar con una plataforma web que almacene datos históricos de las carreras, es disminuir el 
			trabajo que poseen estos departamentos al momento de requerir alguna información curricular.

			