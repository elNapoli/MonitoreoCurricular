\subsection{Arquitectura MVC?}

\subsection{Tecnologías para el desarrollo de la plataforma}

\subsubsection{Front-end}

Esta capa es la parte del software que interactúa con el o los usuarios, en ésta  se encuentran todas las tecnologías que corren del lado del cliente, es decir, todas aquellas tecnologías que corren del lado del navegador web. Es por ello que es de vital importancia de que el front-end sea capaz de entregar  al usuario todas las herramientas necesarias para que éste pueda realizar una correcta interacción con el sistema. Entre las tecnologías usadas en esta capa se encuentran las habituales en el desarrollo Web, tales como HTML, CSS y JavaScript junto con otras que se describirán a continuación.

\myparagraph{JQuery}

JQuery es una librería JavaScript rápida, liviana y con amplias funcionalidades. Hace mucho más simple tareas como recorrer y manipular un documento HTML, manejar eventos, animaciones, e interacciones Ajax	\footnote{Ajax: Asynchronous JavaScript And XML} por medio de una API fácil de usar que funciona a través de múltiples navegadores. Con una combinación de versatilidad y capacidad de ampliación, esta librería busca cambiar la forma en que las personas escriben JavaScript \cite{JQu15}.
\\

Si bien existen otras librerías de JavaScript (Prototype, MoonTools, entre otros). Se decidió usar JQuery en el proyecto por los siguientes motivos:
\begin{itemize}
	\item Es de uso general, por lo que posee una comunidad activa y una extensa documentación.
	\item Posee una amplia variedad de complementos que facilitan el desarrollo.
	\item Es modular.
	\item Es compatible con todos los navegadores existentes.
\end{itemize}


\myparagraph{Alertify}

Alertify es un script escrito con Jquery, el cual nos permite utilizar los siguientes elementos Javascript personalizados: alert(), confirm() y prompt(). Además también nos permite utilizar sus notificaciones, las cuales son muy agradables y sencillas de utilizar y modificar\cite{ALE15}.
\\

Alertify ha sido construido para personalizar nuestras alertas y notificaciones, de esta manera el front-end de la plataforma es mas amigable al usuario y esto permite un mejor entendimiento de los eventos que se realizan en tiempo de ejecución. Además es un plugin multi-idioma y posee responsive-design \footnote{ \textbf{ Responsive Design} es un nuevo paradigma del desarrollo web. Permite adaptar cada sitio a los diferentes formatos de dispositivos de acceso; smartphones, tabletas, portátiles, etc.}
\\

A pesar de que la mayoría de los plugins de notificaciones presentan inconvenientes al momento de implementar con vb.net, se decidió usar un sistema de alertas principalemente para facilitar la visualización de todos los eventos que ocurren en el sistema.


\myparagraph{Boostrap}

Boostrap fue creado a mediados del 2010 por un diseñador y un desarrollador de la red social Twitter, es un proyecto de código abierto, y es uno de los frameworks de front-end más populares en el mundo. Sirvió como guía de estilo para el desarrollo de herramientas internas en la empresa durante más de un año antes de su lanzamiento público, y continua haciéndolo hoy en día\cite{boo15}.
\\

El uso de un framework hace posible que el desarrollo del front-end sea: Fácil, ya que la mayoría de los framework posee una curva de aprendizaje baja, es decir, poseen una gran eficiencia en el  aprendizaje, lo que permite dominar la mayoría de los componentes en un tiempo reducido; Optimizado para dispositivos móviles, puesto que bootstrap posee todas las reglas CSS necesarias para hacer que los sitios se  adapten dinámicamente a la gran mayoría de pantallas y resoluciones existentes en el mercado.


\myparagraph{Template Metis}

Un template es un conjunto de archivos que determinan la estructura y el aspecto visual de un sitio web, y tiene como ventaja principal disminuir tiempos y costos de desarrollo \cite{gli15}. 
\\

El uso de un template disminuye el tiempo de desarrollo de un diseño web, por lo que el alumno tesista se puede centrar en la funcionalidad del sistema que es lo principal."Metis es una  template de administración gratuito basado en Twitter Bootstrap 3.x"\cite{git15}, fue diseñado por un usuario de gitHub y lo puso a disposición para que cualquier usuario lo pueda utilizar.

\myparagraph{Parsley}



\subsubsection{Back-end}
\subsubsection{capa de datos}
