	\documentclass[12pt]{article}
\usepackage[utf8]{inputenc}
\usepackage{anysize} 
\usepackage{hyperref}
\usepackage{graphicx}
\usepackage{multirow}
\usepackage{longtable} % para tablas largas
%palabras y su separación 

\hyphenation{res-pecto}
\hyphenation{re-ferente} 
\hyphenation{desa-rrollo}
\hyphenation{mode-lo}
\hyphenation{requeri-mientos}



\usepackage[x11names,table]{xcolor}
\usepackage{color, colortbl}
%\marginsize{2cm}{0cm}{3cm}{1cm} % Controla los márgenes {izquierda}{derecha}{arriba}{abajo}. 
\renewcommand{\figurename}{Figura}
\renewcommand{\refname}{Referencias}
\renewcommand{\contentsname}{Índice}
\renewcommand\tablename{Tabla}
\hypersetup{pdftitle={Archivos PDF},colorlinks=true,%
pdfstartview=Fit,pdfview=Fit,linkcolor=blue}

\begin{document} 
  \newcolumntype{g}{>{\columncolor{gray}}p}
    \begin{tabular}{ |g{3cm}|p{13cm}|} \hline
	\rowcolors{1}{}{gray!20}
	& \includegraphics[width=3in]{logo_uach_informatica.png} \\
	&  \\
	& \\
	&  \begin{center}  {\Large {\bf Primer Semestre 2015}} \end{center}	\\
	& \\
	& \\ 
	&  \begin{center}  {\LARGE {\bf Sistema de apoyo al monitoreo curricular de pregrado UACh}} \end{center}	\\
	& \\
	& \\ 
	&  \begin{center}  {\Large {\bf Baldomero Águila Napoli}} \end{center}	\\
	& \\
	& \\ 
	&  \begin{center}  {\Large {\bf \parbox[c]{7cm}{\centering Mauricio Ruiz-Tagle\\	 Patrocinante} }} \end{center}	\\
	& \\
	& \\ 
	& \\

	& \\

	& \begin{center}  {\small {\bf Valdivia, Abril de 2015 }} \end{center} \\ \hline
	\end{tabular}
	\thispagestyle{empty}
	\newpage
	\thispagestyle{empty}
	\tableofcontents
	\newpage
  \setcounter{page}{1}
	
	
	\section{PRESENTACIÓN GENERAL}
		\subsection{Nombre del Proyecto}
			\begin{tabular}{ |c|} \hline
			\\
			 {\large Sistema de apoyo al monitoreo curricular de pregrado UACh}\\
			\\
			 \hline
			
			\end{tabular}
		 
		\subsection{Dominio}
			\begin{tabular}{ |c|} \hline
			\\
			 {\large Informática, Tecnología}\\
			\\
			 \hline
			 \end{tabular}
		\subsection{Área de Aplicación}
			\begin{tabular}{ |c|} \hline
			\\
			 {\large Administración Académica}\\
			\\
			 \hline
			 \end{tabular}
		\subsection{Duración del proyecto}
		
			\begin{tabular}{ |c|c|c|} \hline
			& & \\
			 0 & 8 & {\large Meses }\\
			
			& & \\
			 \hline
			
			\end{tabular}
			\newpage
	\section{RESPONSABLES DEL PROYECTO}
		\subsection{Institución principal del proyecto}
		\begin{large}
			\begin{tabular}{|l|l|}
				\hline
				
					\multicolumn{2}{|p{15cm}|}{ \parbox[t]{15cm}{ {\bf Nombre de la Institución} \\	 Departamento de Aseguramiento de la Calidad e Innovación Curricular (DACIC)}}\\ 
				\hline
				{ \parbox[t]{8cm}{ {\bf Dirección} \\Edificio Vicerrectoría Académica · Campus Isla · Teja · Av. Carlos Ibáñez del Campo}} & { \parbox[t]{8cm}{ {\bf Ciudad} \\	Valdivia}}
		\\ 
		
				\hline
				{ \parbox[t]{8cm}{ {\bf Teléfono} \\	+56 63 2221085}} & { \parbox[t]{8cm}{ {\bf E-mail} \\  dacic@uach.cl}}
		\\ 
		
				\hline

			\end{tabular}
		\end{large}




		\subsection{Patrocinante del proyecto}
		\begin{large}
			\begin{tabular}{|l|l|}
				\hline
				
			{ \parbox[t]{7cm}{ {\bf Nombre Completo} \\	Dr. Jorge Mauricio Ruiz-Tagle Molina}}& { \parbox[t]{7cm}{ {\bf R.U.T.} \\	8.214.713-6}}\\ 
				\hline
				{ \parbox[t]{8cm}{ {\bf Dirección} \\	Campus Isla Teja, Valdivia}} & { \parbox[t]{8cm}{ {\bf Ciudad} \\	Valdivia}}
		\\ 
		\hline
							\multicolumn{2}{|p{15cm}|}{ \parbox[t]{15cm}{ {\bf Cargo Actual} \\ Director de estudios de Pregrado }}\\ 
							
				\hline
				{ \parbox[t]{8cm}{ {\bf Teléfono} \\ (56-63) 2221259	}} & { \parbox[t]{8cm}{ {\bf E-mail} \\ mruiztag@uach.cl}}
		\\ 
		
				\hline

			\end{tabular}
		\end{large}
	

		\subsection{Datos del estudiante}
		\begin{large}
			\begin{tabular}{|l|l|}
				\hline
				
			{ \parbox[t]{7cm}{ {\bf Nombre Completo} \\Baldomero Águila Napoli}}& { \parbox[t]{7cm}{ {\bf R.U.T. y Firma} \\	17.536.925-2}}\\ 
				\hline
				{ \parbox[t]{8cm}{ {\bf Dirección} \\	General Lagos 1946}} & { \parbox[t]{8cm}{ {\bf Ciudad} \\	Valdivia}}\\ 

							
				\hline
				
				{ \parbox[t]{8cm}{ {\bf Teléfono} \\ +569-61729928}} & { \parbox[t]{8cm}{ {\bf E-mail} \\ baldomero.napoli@gmail.com}}
		\\ 
		
				\hline

			\end{tabular}
		\end{large}
		\newpage
	\section{RESUMEN DEL PROYECTO}
		
			"La Universidad Austral de Chile es una institución acreditada que forma profesionales y graduados de pre y postgrado, con un sello caracterizado por la excelencia académica, el compromiso con la libertad y con el medio sociocultural, el respeto por la diversidad, la responsabilidad social, entre otros"\cite{MOD07}. Es por esto mismo que es de gran importancia el conocer el historial curricular de cada carrera.
			\\
			
			El objetivo principal del presente proyecto de tesis consiste en diseñar y desarrollar un prototipo de plataforma web que permita gestionar el historial curricular de cada carrera de la Universidad Austral de Chile, el cual permitirá a distintas Unidades de la universidad tener una mejor información curricular de las carreras y así facilitar el trabajo que día a día realizan.
			\\

			
			El sistema web se desarrollará para las dependencias de la Universidad Austral de Chile, es por eso mismo que el alumno tesista debe adaptarse a las tecnologías que la universidad utiliza, por esta razón la solución se desarrollará en las siguiente tecnologías: Microsoft Visual Studio 2013, Microsoft SQL Server 2008 (solo en ambiente de desarrollo, una vez finalizado el proyecto se migrará a SyBase, el cual es el motor de base de datos que utiliza la universidad), Visual Basic como lenguaje del servidor, JavaScript, css3, HTML5 y Tortoise .
			\\

			
			El sistema web beneficiará a los departamentos del área de pregrado de la Universidad en cual están constantemente manipulando información curricular de las carreras, estos departamento son los siguientes: Departamento de Aseguramiento de la Calidad e Innovación Curricular (DACIC), Departamento de Registro Académico Estudiantil y  Departamento de Admisión y Matricula . Las ventajas de contar con una plataforma web que almacene datos históricos de las carreras, es disminuir el trabajo que poseen estos departamentos al momento de requerir alguna información curricular.

			
			


\newpage
\section{OBJETIVOS GENERALES Y ESPECÍFICOS}
		\subsection{Objetivo General}
			{\large  Diseñar y construir  un prototipo de una plataforma web que apoye al monitoreo curricular de pregrado UACh.}
		\subsection{Objetivos Específicos}
		\begin{large}
			
			\begin{itemize}
				\item Conocer cómo los distintos departamentos que integran el departamento de estudios de pregrado (DEP) se enfrentan a los distintos cambios que se producen en los planes de estudio, con el fin de poder entender el contexto en el cual se desarrollará la Base de datos.
				\item  Definir requerimientos del sistema, describiendo sus funcionalidades y separar en módulos la aplicación.
				\item Diseñar e implementar el módulo necesario que permita desplegar el historial curricular de una carrera en particular.
				\item Diseñar e implementar el módulo necesario que permita desplegar el historial de la escuela de una carrera en particular.
				\item Realizar pruebas de validación de los requisitos y estabilidad del prototipo de
plataforma web.
			\end{itemize}
		\end{large}
		\newpage
		
	\section{DESCRIPCIÓN DEL PROYECTO}
	
		\subsection{Introducción}
	  El un plan de estudio es un documento académico el cual sirve para  gestionar una carrera profesional en una universidad.
	  
	  que debe asegurar la formacion profesional 
de calidad de sus estudiantes en una determinada especialidad.
     	 \newpage
     	 
     	 
    

		\subsection{Nivel actual}
	
\newpage
		\subsection{Motivación}
		\begin{itemize}
		 \item Aportar a la Universidad Austral de Chile con un software el cual resuelva la problemática de la poca 
		 documentación en los cambios Curriculares que se efectuan anualmente.
		 \item 

		\end{itemize}

 
		\subsection{Impactos}

\section{RESULTADOS VERIFICABLES RELACIONADOS CON LOS  OBJETIVOS ESPECÍFICOS DEL \\ PROYECTO}

			\begin{tabular}{ |p{15cm}|} \hline
				 \parbox[c]{15cm}{ {\bf Objetivo específico:}\\ \\Conocer cómo los distintos departamentos que integran el departamento 
				 de estudios de pregrado (DEP) se enfrentan a los distintos cambios que se producen en los planes de estudio, con 
				 el fin de poder entender el contexto en el cual se desarrollará la Base de datos.\\} 
			\\
			 \hline
				 \parbox[c]{15cm}{ {\bf Descripción del resultado:}\\ 
				 
				 Documento que contenga: \\
				 \begin{itemize}
				  \item Departamentos que tienen relación directa con el sistema a desarrollar.
				  \item Descripción de las funciones de cada departamento.
				  \item Modelo entidad Relación
				 \end{itemize}
				  Software: \\
				  \begin{itemize}
				   \item Base de datos creada en sql.
						\item Base de datos parcialmente \textit{poblada}.
				  \end{itemize}

			 
				  } 
			 \\ \hline			 
			 \end{tabular}	
			 \\ \\  \\ 
			\begin{tabular}{ |p{15cm}|} \hline
				 \parbox[c]{15cm}{ {\bf Objetivo específico:}\\ \\ Definir requerimientos del sistema, describiendo 
				 sus funcionalidades y separar en módulos la aplicación.\\} 
			\\
			 \hline
				 \parbox[c]{15cm}{ {\bf Descripción del resultado:}\\ 
				 
			Documento que contenga: \\
				 \begin{itemize}
				  \item Especificación  de requisitos funcionales y no funcionales.
				  \item Identificación los módulos del software.
				
				 \end{itemize}
			
				 
				 
				 } 
			 \\ \hline			 
			 \end{tabular}	
			 \\ \\  \\ 
			 \begin{tabular}{ |p{15cm}|} \hline
				 \parbox[c]{15cm}{ {\bf Objetivo específico:}\\ \\Diseñar e implementar el módulo necesario que permita desplegar el historial curricular de una carrera en particular.\\} 
			\\
			 \hline
				 \parbox[c]{15cm}{ {\bf Descripción del resultado:}\\ 
				 
			Documento que contenga: \\
				 \begin{itemize}
				  \item Artefactos UML del módulo: Diagrama de casos de uso, casos de uso, diagrama de complemento, diagramas de secuencia.
				  \item Mockups de la plataforma.
				
				 \end{itemize}
			Software con las siguientes funcionalidades:\\
			      \begin{itemize}
							\item Estructura principal del proyecto.
							\item Conexión con la base de datos.
			       \item \textit{Layout} principal del sistema.
						\item Programado los procedimientos almacenados que tengan relación con el módulo.
			     
			      \end{itemize}

				 
				 
				 } 
			 \\ \hline			 
			 \end{tabular}	
			 \\ \\  \\ 
			\begin{tabular}{ |p{15cm}|} \hline
				 \parbox[c]{15cm}{ {\bf Objetivo específico:}\\ \\ Diseñar e implementar el módulo necesario que permita desplegar el historial de la escuela de una carrera en particular.\\} 
			\\
			 \hline
				 \parbox[c]{15cm}{ {\bf Descripción del resultado:}\\ 
				 Documento que contenga: \\
				 \begin{itemize}
				  \item 
				 \end{itemize}

				 
					} 
			 \\ \hline			 
			 \end{tabular}	
			 \\ \\  \\ 
			\begin{tabular}{ |p{15cm}|} \hline
				 \parbox[c]{15cm}{ {\bf Objetivo específico:}\\ \\ Diseñar una base de datos adaptada a la información manejada en contextos técnico-tácticos de equipos deportivos.\\} 
			\\
			 \hline
				 \parbox[c]{15cm}{ {\bf Descripción del resultado:}\\ 
				 
Generar un documento formal que contenga lo siguiente:			 \\ 
\begin{itemize}
	\item identificación del motor l base de datos a utilizar.
	\item Estructura de los datos.
	\item Modelo Entidad Relación.
	\item Restricciones.
\end{itemize}	
				  } 
			 \\ \hline			 
			 \end{tabular}	
			 \\ \\  \\ 
			\begin{tabular}{ |p{15cm}|} \hline
				 \parbox[c]{15cm}{ {\bf Objetivo específico:}\\ \\Diseñar una interfaz gráfica para el ingreso de datos y visualización de resultados, tanto de forma gráfica como tabular. \\} 
			\\
			 \hline
				 \parbox[c]{15cm}{ {\bf Descripción del resultado:}\\
				 
				  Prototipos de la aplicación con sus funcionalidades completas.} 
			 \\ \hline			 
			 \end{tabular}	
			 \\ \\  \\ 
			\begin{tabular}{ |p{15cm}|} \hline
				 \parbox[c]{15cm}{ {\bf Objetivo específico:}\\ \\Validar la solución, tanto desde aspectos técnicos como de usabilidad. \\} 
			\\
			 \hline
				 \parbox[c]{15cm}{ {\bf Descripción del resultado:}\\ \\
					Prototipo finalizado y encuesta de validación.				 
				  \\} 
			 \\ \hline			 
			 \end{tabular}	
	\newpage

			\bibliography{Anteproyecto}
			\bibliographystyle{alpha}
\end{document}
